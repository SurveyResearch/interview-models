\documentclass[11pt,twoside]{article}
\usepackage[toc,page,header]{appendix}
\usepackage{pdfpages}
\usepackage{csquotes}
\usepackage{changepage}
\usepackage{fontspec}
\defaultfontfeatures{Scale=MatchLowercase}
\setmainfont[Mapping=tex-text]{Times New Roman}
\setsansfont[Mapping=tex-text]{Arial}
\setmonofont{Courier}

\usepackage{float}
\usepackage{turnstile}
\usepackage{bussproofs}

\usepackage{geometry}
\geometry{letterpaper}

\newtheorem{theorem}{Theorem}
%\newtheorem{cor}{Corollary}
%\newtheorem{lem}{Lemma}
%\theoremstyle{remark}
\newtheorem{remark}{Remark}

\newtheorem{objection}{Objection}
\newenvironment*{response}[1][]{\noindent
\textbf{Response to Objection #1.}
\begin{adjustwidth}{1em}{1em}
}
{\end{adjustwidth}
\vspace{1ex}
}


%\usepackage[parfill]{parskip}    % Activate to begin paragraphs with an empty line rather than an indent

\usepackage{graphicx}
\usepackage[leftcaption]{sidecap}
\sidecaptionvpos{figure}{c}

%\usepackage{amssymb}

\usepackage{epstopdf}
\DeclareGraphicsRule{.tif}{png}{.png}{`convert #1 `dirname #1`/`basename #1 .tif`.png}

\usepackage[
bibstyle=numeric,
citestyle=authortitle,
natbib=true,
hyperref,bibencoding=utf8,backref=true,backend=biber]{biblatex}

\usepackage{hyperref}
\hypersetup{
    bookmarks=true,         % show bookmarks bar?
    unicode=true,          % non-Latin characters in Acrobat’s bookmarks
    pdftoolbar=true,        % show Acrobat’s toolbar?
    pdfmenubar=true,        % show Acrobat’s menu?
    pdffitwindow=false,     % window fit to page when opened
    pdfstartview={FitH},    % fits the width of the page to the window
    pdftitle={The Survey Interview: Models and Protocols},    % title
    pdfauthor={G. A. Reynolds},     % author
    pdfsubject={Survey Interviews},   % subject of the document
    pdfcreator={G. A. Reynolds},   % creator of the document
    pdfproducer={Producer}, % producer of the document
    pdfkeywords={keyword1} {key2} {key3}, % list of keywords
    pdfnewwindow=true,      % links in new window
    colorlinks=true,       % false: boxed links; true: colored links
    linkcolor=blue,          % color of internal links
    citecolor=blue,        % color of links to bibliography
    filecolor=magenta,      % color of file links
    urlcolor=cyan           % color of external links
}
\usepackage{draftwatermark}


\usepackage{fancyhdr}
\setlength{\headheight}{15.2pt}
\pagestyle{fancy}

\lhead[Models of Interviewing]{\thepage}
\chead[]{}
\rhead[\thepage]{Models of Interviewing}

\title{Models of Survey Interviewing}
\author{G. A. Reynolds}
\date{\today}
\bibliography{%
../bib/abstracts.bib,%
../bib/brandom.bib,%
../bib/causality.bib,%
../bib/em.bib,%
../bib/logic.bib,%
../bib/mind.bib,%
../bib/philosophy.bib,%
../bib/pragmatism.bib,%
../bib/psychomet.bib%
../bib/psychometrics.bib,%
../bib/misc.bib,%
../bib/measurement.bib,%
../bib/psychology.bib,%
../bib/variables.bib,%
../bib/val.bib,%
../bib/validity.bib,%
}

%% Macros

\newcommand{\SM}{Standard Model}
\newcommand{\XSM}{Extended Standard Model}

\newcommand{\SMeth}{Survey Methodology}

\newcommand{\SR}{Survey Research}
\newcommand{\sr}{survey research}
\newcommand{\SRIV}{Survey Interview}
\newcommand{\sriv}{survey interview}
\newcommand{\SIV}{Survey Interviewing}
\newcommand{\FI}{Field Interviewer}
\newcommand{\Iver}{Interviewer}
\newcommand{\R}{Respondent}
\newcommand{\LPR}{Legal Permanent Resident}
\newcommand{\ART}{Assimilated Response Technique}
\newcommand{\GAM}{Grouped Answer Method}
\newcommand{\IOM}{Instrument of Measurement}

\includeonly{%
%% pilots,cards
}
%%%%%%%%%%%%%%%%%%%%%%%%%%%%%%%%%%%%%%%%%%%%%%%%%%%%%%%%%%%%%%%%
\begin{document}
\maketitle
\nocite{*}

\begin{abstract}
This paper analyzes and compares three models of survey
interviewing.  That is, models of the conduct of survey interviewing,
rather than models of the structure of questionnaires, interviews,
etc.  The first is the Laboratory Model, which is motivated by a
desire to mimic the experimental physical sciences, paradigmatically
physics.  The paradigmatic example of this sort of model is the
``Standardized Survey Interview''.  Analysis of this model exposes a
variety of (usually) unacknowledged commitments to
theoretical/philosophical doctrines, which are shown to be untenable.

The second model is the Extended Laboratory Model.  This is a
modification of the Laboratory Model.  It acknowledges that, due to
the interactive nature of the interview, the interviewer inevitably
makes a contribution.  But it retains the basic structural commitments
of the laboratory Model.  An example of an Extented Laboratory model
is Maynard et al's ``alternating model''.

The third model is The Theatrical Model.  This model is similar to the
Laboratory Model, in that it recommends that the interviewer read the
questions exactly as written, avoid probes, etc., but it involves a
very different conceptualization of the nature of interviewing.  Like
the Extended Laboratory Model, it acknowledges that the Field
Interviewer makes a substantial contribution to the survey interview,
due to the fundamentally interactive and collaborative nature of
discursive practice.  But it stresses that interviewing essentially
involves role-playing.  This model is based on a more realistic
picture of the nature of surveys and survey interviewing, but it also
has some weaknesses, which we analyze.

Finally, the fourth model is The Collaborative Model.  This model is
driven by a closer and more realistic analysis of the nature of the
survey interview.  It demystifies aspects of the interview that the
other two models take for granted or ignore, such as the various
asymmetries involved in interviews, the fact that completion of a
survey questionnaire is the joint responsibility of the interviewer
and the respondent, and so forth.  It discards the fictions that are
at the core of the other models discussed.  Most critically, motivated
by considerations of the nature of discursive practice and the
production of meaning, it denies that survey interviewing involves
measurement.  In summary, this model recommends that survey
interviewing be construed as collaborative or joint action, and that
the demystified facts of the matter be openly acknowledged in the
conduct of interviews.  This means, among other things, that the field
interviewer should serve as an assistant to the respondent, rather
than a proxy for the researcher; that interviewer and respondent are
jointly responsible for completing the questionnaire; and that the
results of individual survey interviews should be viewed as a trace a
kind of dialog between the individuality of the particular respondent
and the stereotype presupposed by the questionnaire design.
\end{abstract}

\tableofcontents
\listoffigures

\newpage
%%%%%%%%%%%%%%%%%%%%
\section{The Laboratory Model}

\begin{remark}
  TODO: the legitimacy of this model is based on theoretical
  (philosophical) commitments that are usually accepted uncritically
  (since they are virtually never explicitly even acknowledged).  The
  organization of this section thus should be: describe the model
  using Standardized Interviewing as the paradigm; show that it is
  motivated by a felt need to mimic experimental physical science;
  show how adoption of this model implicitly entails commitment to
  various theoretical/philosophical doctrines (theories, dogmas) as
  models of various aspects of the Survey Interview; critique the
  adequacy of these implicit underlying doctrines as models of
  interviewing.
\end{remark}

The Laboratory Model is driven by the desire to mimic the techniques
and methods of the physical sciences, paradigmatically physics.  It is
implicitly committed to the following presuppositions:

\begin{itemize}
\item Questions are instruments of measurement
\item The Field Interview is a laboratory technician
\item The Respondent is a laboratory sample
\item etc.
\end{itemize}

The Standardized Survey Interview is the paradigmatic example of this
sort of model.  The presuppositions listed above are implemented in
the basic recommendations of this model:

\begin{itemize}
\item Read the question exactly as worded
\item Do not probe
\item etc.
\end{itemize}

\subsection{Critique}

There are many problems with this model, some of which have been
explored by researchers focussing on the interactive nature of the
interview.

First, this model idealizes the FI as a kind of robot; ideally, FIs
are interchangeable parts.  Of course even those researchers most
committed to this model acknowledge that it is impossible to attain
this ideal in practice.  The critical question is whether it is even
desireable (rational) to adopt this model as an ideal.  It is not only
unrealistic as a practical matter; even theoretically, this model is
badly motivated.

Second, under this model the ideal FI makes no contribution to the
interview; all the work is done by the text of the question and the
Respondent.  Not only are FIs not robots, the notion that one human
being can ask a question of another human being without making a
substantial contribution to the exchange is clearly false.  Even if we
could construct interviewing robots, the dynamics of asking and
answering questions would require

A more formal way of stating this point is to observe that the ideal
is based on an implicit premise of independence.  The idea, clearly
drawn from experimental physical sciences, is that the actions of the
experimenter merely serve to trigger the causal chain the results in
the experimental outcome, which is independent of the ``quality'' of
the experimenter's actions.  In other words, the ideal is that the
actions of experimenter should not be genuine \textit{actions} at all;
rather, they should be steps in an \textit{effective procedure}, and an
effective procedure is one that can be performed by a mindless robot.

\begin{remark}
  Note the connection between scientific experimental method and the
  concept of ``effective procedure''.  The latter is the fundamental
  concept of computation; it only arose in the late 19th century in a
  strictly mathematical context, and was given definitive explicit
  meaning by Turing in 1936.  But experimental method implicitly
  involved a notion of effective procedure long before this notion
  emerged explicitly in mathematics.
\end{remark}

The SSIM adopts this perspective with respect to the interview, and
thus presupposes that the asking and answering of questions are
independent actions.  But this is clearly not the case, even in the
ideal situation.

\begin{remark}
  TODO: trace this to underlying philosophical commitments.  In the
  case of SSIM, the underlying commitment is to semantic atomism
  [check terminology -GR].  In other words, once the researcher
  decides that the interview should be modeled on the physical
  experiment, it remains to justify this, to show that is is a
  ``good'' model for survey interviewing.  The major philosophical
  (theoretical) resources available to support this model are:
  semantic atomism, the telementation model of communication, etc.
\end{remark}

Third, this model fails to recognize that rules are not sufficient; in
order to successfully ``administer'' an interview, the FI must cope
with the specific circumstances of particular interviewing situations,
and must decide in each case exactly how to apply the rules of the
SSIM.  By now it is well known both theoretically (Wittgenstein et
al.) and empirically (Ethnomethodolgy, Conversation Analysis) that
rules are not enough; to apply a rule, individuals inevitably must
improvise in order to accomodate local contingencies.  Researchers in
\SR{} [TODO: citations] have studied in some detail how FIs do this in
actual interviews even when they attempt to follow the rules of SSIM.
Such empirical research has clearly shown what theoretical
considerations also demonstrate, namely the the SSIM idealization of
the FI is a bad model.

Fourth, and perhaps most important (and most problematic), this model
presupposes that questions and questionnaires are instruments of
measurement, and that asking a question and obtaining an answer
amounts to taking a measurement.

\begin{remark}
  TODO: flesh this out.  A detailed analysis of the notion of
  questions as measurements is beyond the scope of this paper, but we
  need to say something at least about what sort of implicit
  theoretical doctrines are implicated by this idea, and why they
  fail.
\end{remark}

%%%%%%%%%%%%%%%%
\section{The Extended Laboratory Model}

%%%%%%%%%%%%%%%%
\section{The Theatrical Model}

The theatrical model adopts some of the rules of the SSIM, but
discards the idealized laboratory model of interviewing.  It
acknowledges that the FI inevitably makes a substantial contribution
to the interview.  Instead of modeling the FI on an idealized
laboratory technician or apparatus, it relies on the analogy between
an FI reading a question to a Respondent and the performance of an
actor before an audience.  The questionnaire (written by the
Researcher) is analogous to a script; the FIs job is play the role
defined by the script, just like an actor on stage.

Instead of ``read the question exactly as worded'', the instruction to
the FI is the director's plea: stay on script.  Do not ad lib.

This model acknowledges that the script alone is not sufficient to the
performance.  Like an a playscript, a questionnaire can only come to
life in the hands of a performer, whose work cannot be antecedently
specified in full.  Good theatrical directors know better than to try
to tell an actor precisely \textit{how} to deliver a line; figuring
that out is the responsibility of the actor and only the actor.
Paradoxically, good acting always involves a substantially element of
spontanaeity, even where the performance is tightly scripted.  Replace
the actor's spontenaeity by the director's procedure and you kill the
performance.

The same goes for survey interviewing.

\begin{remark}
An important element of this model is the observation that, at least
in many cases, the FI effectively acts as the Researcher's mouthpiece.
The FI seems to ask questions, but this is an illusion (just like a
stage play); in fact, it is not the FI asking the question, it is the
Researcher.  And the both the FI and Respondent know this.  At best
the FI may be said to ask questions on behalf of the Researcher; at
worst, the FI cannot be said to be genuinely asking questions at all,
any more than a Hamlet can be said to genuinely die at the end of the
play.
\end{remark}

\begin{remark}
  TODO: flesh out the model and critique it.  What are the underlying
  commitments?  What about measurement?
\end{remark}

\subsection{Critique}

%%%%%%%%%%%%%%%%
\section{The Collaborative Model}

This model may be thought of as a kind of de-mystified model of the
interview.  Unlike the theatrical model, this model asks the FI and R
to explicitly acknowledge the various fictions and positionings
involved in a survey interview and go from there.

The role of the FI in this model is neither to merely execute an
effective procedure (as in the SSIM), nor to play a role in a
semi-fictional drama, as in the theatrical model.  Instead, the FI is
asked to collaborate with the Respondent in order to complete the
questionnaire.

\begin{remark}
  The main recommendation for this model is that it is based on a more
  realistic assessment of the nature of survey interviews.
\end{remark}

This model explicitly acknowledges:

\begin{itemize}
\item The FI asks questions on behalf of a Researcher
\item Questions are ``recipient designed'' (Bakhtin), but the
  recipient is a stereotype: the Ideal Respondent
\item The survey interview is asymmetric (epistemic, power, control of
  flow, stereotypical v. actual respondent etc.)
\item Questions and answers are discursive moves in the ``space of reasons''
\item Both FI and Respondent must improvise (in CA-speak, ``produce local order'' etc.)
\item No measurement (properly so-called) is involved
\item Completion of the interview (questionnaire) is the joint
  responsibility of both the FI and the Respondent.
\end{itemize}

\begin{remark}
  ``Stereotype design'' - in constructing the questionnaire, the
  Respondent inevitably must have some idea of who the respondent(s)
  might be.  This is not (\textit{pace} Houtkoop-Steenstra) a matter
  of ``audience'' design as opposed to ``recipient'' design; audiences
  do not answer questions.  The relevant distinction is between design
  for a particular (specific individual respondent) v. design for the
  general (idealized or ``(stereo-) typical'' member of the research
  population); hence ``stereotype design''.  [NB: Putnam on concepts
    as stereoptypes, \cite{putnam_meaning_1975}]

  This has rarely (never to my knowledge) been recognized, but it has
  major implications.  It means that the whole show is biased from the
  very beginning by the Researcher's notion of what counts as a
  stereotypical member of the research population.  And again, this
  source of bias is never (to my knowledge) explicitly addressed.  But
  the stereoptype, since it drives the design of the questionnaire,
  should itself be an antecedent topic of research.

Alternatively, the survey can itself be viewed as an investigation
into the character of the stereotypic member.  The ``errors'' should
be viewed not as errors, but as variations that may motivate a
revision of the stereotype.

\end{remark}

\clearpage
\appendix
\begin{appendices}
\section{Bibliography}
%% \addcontentsline{toc}{chapter}{Bibliography}
%% \bibliographystyle{plainnat}
%% \printbibliography[heading=none]
\end{appendices}

\end{document}
